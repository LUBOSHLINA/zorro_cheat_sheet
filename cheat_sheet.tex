%%%%%%%%%%%%%%%%%%%%%%%%%%%%%%%%%%%%%%%%%%%%%%%%%%%%%%%%%%%%%%%%%%%%%%
% writeLaTeX Example: A quick guide to LaTeX
%
% Source: Dave Richeson (divisbyzero.com), Dickinson College
% 
% A one-size-fits-all LaTeX cheat sheet. Kept to two pages, so it 
% can be printed (double-sided) on one piece of paper
% 
% Feel free to distribute this example, but please keep the referral
% to divisbyzero.com
% 
%%%%%%%%%%%%%%%%%%%%%%%%%%%%%%%%%%%%%%%%%%%%%%%%%%%%%%%%%%%%%%%%%%%%%%
% How to use writeLaTeX: 
%
% You edit the source code here on the left, and the preview on the
% right shows you the result within a few seconds.
%
% Bookmark this page and share the URL with your co-authors. They can
% edit at the same time!
%
% You can upload figures, bibliographies, custom classes and
% styles using the files menu.
%
% If you're new to LaTeX, the wikibook is a great place to start:
% http://en.wikibooks.org/wiki/LaTeX
%
%%%%%%%%%%%%%%%%%%%%%%%%%%%%%%%%%%%%%%%%%%%%%%%%%%%%%%%%%%%%%%%%%%%%%%

\documentclass[10pt,landscape]{article}
\usepackage{amssymb,amsmath,amsthm,amsfonts}
\usepackage{multicol,multirow}
\usepackage{calc}
\usepackage{graphicx}
\usepackage{color}
\usepackage{listings}
\usepackage{ifthen}
\usepackage[landscape]{geometry}
\usepackage[colorlinks=true,citecolor=blue,linkcolor=blue]{hyperref}


\ifthenelse{\lengthtest { \paperwidth = 11in}}
    { \geometry{top=.5in,left=.5in,right=.5in,bottom=.5in} }
	{\ifthenelse{ \lengthtest{ \paperwidth = 297mm}}
		{\geometry{top=1cm,left=1cm,right=1cm,bottom=1cm} }
		{\geometry{top=1cm,left=1cm,right=1cm,bottom=1cm} }
	}
\pagestyle{empty}
\makeatletter
\renewcommand{\section}{\@startsection{section}{1}{0mm}%
                                {-1ex plus -.5ex minus -.2ex}%
                                {0.5ex plus .2ex}%x
                                {\normalfont\large\bfseries}}
\renewcommand{\subsection}{\@startsection{subsection}{2}{0mm}%
                                {-1explus -.5ex minus -.2ex}%
                                {0.5ex plus .2ex}%
                                {\normalfont\normalsize\bfseries}}
\renewcommand{\subsubsection}{\@startsection{subsubsection}{3}{0mm}%
                                {-1ex plus -.5ex minus -.2ex}%
                                {1ex plus .2ex}%
                                {\normalfont\small\bfseries}}
\makeatother
\setcounter{secnumdepth}{0}
\setlength{\parindent}{0pt}
\setlength{\parskip}{0pt plus 0.5ex}
% -----------------------------------------------------------------------

\begin{document}

\raggedright
\footnotesize



\definecolor{codegreen}{rgb}{0,0.6,0}
\definecolor{codegray}{rgb}{0.5,0.5,0.5}
\definecolor{codepurple}{rgb}{0.58,0,0.82}
\definecolor{backcolour}{rgb}{0.95,0.95,0.92}

\lstdefinestyle{mystyle}{
	backgroundcolor=\color{backcolour},   
	commentstyle=\color{codegreen},
	keywordstyle=\color{magenta},
	numberstyle=\tiny\color{codegray},
	stringstyle=\color{codepurple},
	basicstyle=\footnotesize,
	breakatwhitespace=false,         
	breaklines=true,                 
	captionpos=t,                    
	keepspaces=false,                 
	numbers=left,                    
	numbersep=5pt,                  
	showspaces=false,                
	showstringspaces=false,
	showtabs=false,                  
	tabsize=2 
	%,frame=single
}

\lstset{style=mystyle,language=C}

\begin{center}
     \Large{\textbf{A quick guide to Zorro}} \\
\end{center}
\begin{multicols}{3}
\setlength{\premulticols}{1pt}
\setlength{\postmulticols}{1pt}
\setlength{\multicolsep}{1pt}
\setlength{\columnsep}{2pt}

\section{What is Zorro?}
Zorro is a free trading automation plattform. \href{http://www.opserver.de/ubb7/ubbthreads.php?ubb=cfrm}{Forum}

\section{Variables}

\scalebox{0.55}{
\begin{tabular}{llllll}
Type&Size (bytes)&Range&Step&Digits&Used for\\
double, var&8&-1.8·10308to 1.8·10308&2.2·10-308&~14&prices\\
float&4&-3.4·1038to 3.4·1038&1.2·10-38&~6&storing prices\\
fixed&4&-1048577.999 to 1048576.999&0.001&~9&not for trading\\
int, long&4&-2147483648 to 2147483647&1&~10&counting\\
short&2&0 to 65536&1&~4&not for trading\\
char&1&0 to 256&1&~2&text characters\\
bool&4&true, false&-&-&decisions\\
char*, string&characters+1&"..."&-&-&text\\
var*, vars&elements*8&-1.8·10308to 1.8·10308&2.2·10-308&~14&arrays, series\\
mat&rows*cols*8+12&-1.8·10308to 1.8·10308&2.2·10-308&~14&vectors, matrices
\end{tabular}
}

\section{Skript Control}
 
\begin{lstlisting}[title=if .. else]
if (((x+3)<9) or (y==0))   // set z to 10 if x+3 is below 9, or if y is equal to 0
z = 10;
else    
z = 5;// set z to 5 in all other cases
\end{lstlisting}

\begin{lstlisting}[title=while]
x = 0;
while(x < 100) // repeat while x is lower than 100
x += 1; 
\end{lstlisting}

\begin{lstlisting}[title=for loop]
int i;
for(i=0; i<5; i++) // repeat 5 times
x += i; 
\end{lstlisting}


\begin{lstlisting}[title=switch]
int choice;
...
switch (choice)
{
	case 0:
	printf("Zero! ");
	break;
	case 1:
	printf("One! ");
	break;
	case 2: 
	printf("Two! ");
	break;
	default:
	printf("None of them! ");
}
\end{lstlisting}


\section{Input/Output}




\begin{description} 
\item[printf] (string format, ...) 
\item[print] (int to, string format,.)  
\begin{itemize}
	\item[TO\_WINDOW] print in [Test] and [Trade] mode to the message window and the log file (default for printf).
	\item[TO\_LOG] print in [Test] mode only to the log file, in [Trade] and in STEPWISE mode also to the message window.
	\item[TO\_FILE]
	\item[TO\_ANY]
	\item[TO\_DIAG]
	\item[TO\_CSV]
\end{itemize}   	
\item[msg] (string format, ...): int
\begin{itemize}
	\item[0]  when [No] was clicked,
	\item[1] when [Yes] was clicked.
\end{itemize}
\end{description}


\section{File functions}
\begin{description} 
	\item[file\_copy] (string dest, string src)
	\item[file\_delete] (string name)
	\item[file\_length]
	\item[file\_date]	
	\item[file\_select] Opens a file dialog box at a given directory that lets the user select a file to open. Returns the selected file name including path.
	\item[file\_content]  	\item[file\_read] Reads the content of a file into a string or array. Returns the number of read bytes.
	\item[file\_write] Stores the content of a string, series, or other data array in a file.
	\item[file\_append] Opens a file and appends text or other data to it; can be used to export data to Excel or other programs. If the file does not exist, it is created.
	 
\end{description}

%%%%%%%%%%%%%%%%%%%%%%%%%%%
\section{Testing}
\href{http://zorro-project.com/manual/en/testing.htm}{Testing}

\subsection{Debugging}
if(date() == 20150401) set(STEPWISE);
watch ("!...", ...)

\subsection{Biasis}

Curve fitting bias, Market fitting bias, Peeking bias, Data mining bias, Trend bias, Granularity bias, Sample size bias

\subsection{Asset Parameters}
Spread, eg Spread = 3*PIP
Slippage, default = 10
Commission,  taken from AssetsFix.csv
RollShort
RollLong

%%%%%%%%%%%%%%%%%%%%%%%%%%% code
\section{Code}
\let\clearpage\relax
\section{From Manual}
\subsection{Trade Management}
\begin{lstlisting}[caption=
Chan1ge stops and profit targets of all open long trades with the current algo and asset]
exitLong(0,NewStop);
exitLong(0,-NewTakeProfit);
\end{lstlisting}


\begin{lstlisting}[caption=Limit the number of open positions
// max. 3 open long positions per asset/algo]
#define H24 (1440/BarPeriod)
#define H4 (240/BarPeriod)
#define H1 (60/BarPeriod)

if(my_long_condition == true) {
	exitShort(); // no hedging - close all short positions
	if(NumOpenLong < 3) enterlong();
}
\end{lstlisting}

\begin{lstlisting}[caption=Exit all open trades Friday afternoon GMT]
if(dow() == FRIDAY && hour() >= 18) { 
	exitLong("*");
	exitShort("*");
}
\end{lstlisting}


\begin{lstlisting}[caption=Lock 80\% profit of all winning trades]
for(open_trades)
if(TradeIsOpen && !TradeIsPool && TradeProfit > 0) {
	TradeTrailLock = 0.80; // 80% profit (minus trade costs)
	if(TradeIsShort)
	TradeTrailLimit = max(TradeTrailLimit,TradePriceClose);
	else
	TradeTrailLimit = min(TradeTrailLimit,TradePriceClose);
}
\end{lstlisting}


\begin{lstlisting}[caption=Calculate the value of all open trades with the current asset]
var valOpen()
{
	string CurrentAsset = Asset; // Asset is changed in the for loop
	var val = 0;
	for(open_trades)
	if(strstr(Asset,CurrentAsset) && TradeIsOpen)
	val += TradeProfit;
	return val;
}
\end{lstlisting}

\begin{lstlisting}[caption=Monitoring and modifying a certain trade]

TRADE* MyTrade = enterlong();
...
ThisTrade = MyTrade; // connect trade variables to MyTrade
var MyResult = TradeProfit; // evaluate trade variables
...
exitTrade(MyTrade,0,TradeLots/2); // exit half the trade
} 
\end{lstlisting}

\begin{lstlisting}[caption=Correlation  / heatmap]
#define DAYS  252 // 1 year
#define NN    30  // max number of assets

#include <profile.c>

// plot a heatmap of asset correlations
function run()
{
	BarPeriod = 1440;
	StartDate = 20150101;
	LookBack = DAYS;
	
	vars Returns[NN];
	var Correlations[NN][NN]; // NN*NN matrix
	int N = 0;
	while(asset(loop(.../*some assets*/ ..))) 
	{
		Returns[N] = series((price(0)-price(1))/price(0));
		N++; // count number of assets
	}
	
	int i,j;
	if(!is(LOOKBACK)) {
		for(i=0; i<N; i++)
		for(j=0; j<N; j++)
		Correlations[N*i+j] = Correlation(Returns[i],Returns[j],DAYS);
		
		plotHeatmap(Correlations,N);
		quit("");
	}
}
\end{lstlisting}


\section{Indicators   Signals}

\begin{lstlisting}[caption=DSS bessert]
// http://www.opserver.de/ubb7/ubbthreads.php?ubb=showflat&Main=54970&Number=457800#Post457800
var Periods = 16;
var Smoothing = 4;
vars Price = series(priceClose());

var HiHi = HH(Periods);
var LoLo = LL(Periods);

vars EMA_exp = series((Price[0]-LoLo)/(HiHi-LoLo));

vars Temp = series(EMA(EMA_exp,Smoothing));

var TempHigh = MaxVal(Temp,Periods);
var TempLow = MinVal(Temp,Periods);

vars DSS_exp = series((Temp[0]-TempLow)/(TempHigh-TempLow));

vars DSS_Bressert = series(EMA(DSS_exp,Smoothing));
vars Signal = series(SMA(DSS_Bressert,3));
\end{lstlisting}

\begin{lstlisting}[caption=Generate an indicator with a different asset  time frame  and shift]
//extended ATR function with individual asset and timeframe (in minutes)
var extATR(string symbol,int period,int length,int shift)
{
ASSET* previous = g->asset; // store previous asset
if(symbol) asset(symbol);   // set new asset
if(period) TimeFrame = period/BarPeriod;
// create price series with the new asset / timeframe
vars H = series(priceHigh()), 
L = series(priceLow()),
O = series(priceOpen()),
C = series(priceClose());
TimeFrame = 1; // set timeframe back
g->asset = previous; // set asset back
return ATR(O+shift,H+shift,L+shift,C+shift,length);
}
\end{lstlisting}

\begin{lstlisting}[caption=Calculate the weekend price change for gap trading]
// use 1-hour bars, wait until Sunday Sunday 5pm ET, 
// then get the price change from Friday 5pm ET
if(dow() == SUNDAY && lhour(ET) == 5) { 
	int FridayBar = timeOffset(ET,SUNDAY-FRIDAY,5,0);
	var PriceChange = priceClose(0) - priceClose(FridayBar);
}
\end{lstlisting}


\begin{lstlisting}[caption=Use a series to check if something happened within the last n bars]
// buy if Signal1 crossed over Signal2 within the last 7 bars
...
vars crosses = series(0); // generate a series and set it to 0
if(crossOver(Signal1,Signal2)
crosses[0] = 1; // store the crossover in the series
if(Sum(crosses,7) > 0) // any crossover within last 7 bars?
enterLong();
...
\end{lstlisting}

\begin{lstlisting}[caption=Use a loop to check if something happened within the last n bars]
// buy if Signal1 crossed over Signal2 within the last 7 bars
...
int i;
for(i = 0; i < 7; i++)
if(crossOver(Signal1+i,Signal2+i)) { // crossover, i bars ago?
	enterLong();
	break; // abort the loop
}
...
\end{lstlisting}

\begin{lstlisting}[caption=Align a time frame to a certain event]
// Let time frame start when Event == true
// f.i. frameAlign(hour() == 0); aligns to midnight
function frameAlign(BOOL Event)
{
	TimeFrame = 1;
	vars Num = series(0); // use a series for storing Num between calls
	Num[0] = Num[1]+1;    // count Num up once per bar
	if(!Event) 
	TimeFrame = 0;      // continue current time frame
	else {
		TimeFrame = -Num[0]; // start a new time frame
		Num[0] = 0;          // reset the counter
	}
}
\end{lstlisting}


\begin{lstlisting}[caption=Shift a series into the future]
// the future is unknown, therefore fill 
// all unknown elements with the current value
vars seriesShift(vars Data,int shift)
{
	if(shift >= 0) // shift series into the past
	return Data+shift;
	else { // shift series into the future
		int i;
		for(i = 1; i <= shift; i++)
		Data[i] = Data[0];
		return Data;
	}
}
\end{lstlisting}

\begin{lstlisting}[caption=Use a function from an external DLL]
// Use the function "foo" from the DLL "bar.dll"
// Copy bar.dll into the Zorro folder
int __stdcall foo(double v1,double v2); // foo prototype
API(foo,bar) // use foo from bar.dll

function run()
{
	...
	int result = foo(1,2);
	...
}
\end{lstlisting}

\begin{lstlisting}[caption=Equity curve trading (skipping trades dependent on strategy success)]
// don't trade when the equity curve goes down
// and is below its own lowpass filtered value
function checkEquity()
{
	// generate equity curve including phantom trades
	vars EquityCurve = series(EquityLong+EquityShort);
	vars EquityLP = series(LowPass(EquityCurve,10));
	if(EquityLP[0] < LowPass(EquityLP,100) && falling(EquityLP))
	Lots = -1; // drawdown -> phantom trades
	else
	Lots = 1; // profitable -> normal trades
\end{lstlisting}

\subsubsection{Auxiliary}
\begin{lstlisting}[caption=Debugging a script]
// Display a message box with the variables to be observed
// Click [Yes] for a single step, [No] for closing the box
static int debug = 1;
if(debug && Bar > LookBack)
debug = msg(
"High = %.5f, Low = %.5f",
priceHigh(), priceLow());
\end{lstlisting}

\begin{lstlisting}[caption=Find out if you have a standard  mini  or micro lot account]
// Click [Trade] with the script below
function run()
{
	asset("EUR/USD");
	if(Bar > 0) {
		if(LotAmount > 99999) 
		printf("\nI have a standard lot account!");
		else if(LotAmount > 9999) 
		printf("\nI have a mini lot account!");
		else 
		printf("\nI have a micro lot account!");
		quit();
	}
}
\end{lstlisting}


\begin{lstlisting}[caption=Download historic price data]
// Click [Test] for downloading/updating the latest "NZD/USD" price data
// This extends the length of the historical price data file, therefore
// WFO strategies using that asset should be trained again afterwards
function run()
{
	NumYears = 6;     // download up to 6 years data 
	loadHistory("NZD/USD",1); // update the price history
}
\end{lstlisting}

\begin{lstlisting}[caption=Export historic price data to a .csv file]
// Click [Test] for exporting price data to a .csv file in the Data folder
// The records are stored in the format: time, open, high, low, close 
// f.i. "31/12/12 00:00, 1.32205, 1.32341, 1.32157, 1.32278"
// Dependent on the locale, date and numbers might require a different format
function run()
{
	BarPeriod = 1440;
	StartDate = 2008;
	EndDate = 2012;
	LookBack = 0;
	
	string line = strf(
	"%02i/%02i/%02i %02i:%02i, %.5f, %.5f, %.5f, %.5f\n",
	day(),month(),year()%100,hour(),minute(),
	priceOpen(),priceHigh(),priceLow(),priceClose());
	if(is(INITRUN))
	file_delete("Data\\export.csv");
	else
	file_append("Data\\export.csv",line);
}
\end{lstlisting}
\begin{lstlisting}[caption=Print the description of a trade (like "[AUD/USD:CY:S1234]") in the log]
void printTradeID()
{
	string ls = "L", bo = "[", bc = "]";
	if(TradeIsShort) ls = "S";
	if(TradeIsPhantom) { bo = "{"; bc = "}"; }
	printf("#\n%s%s:%s:%s%04i%s ",
	bo,TradeAsset,TradeAlgo,ls,TradeID%10000,bc);
}
\end{lstlisting}
\begin{lstlisting}[caption=Plot equity curves of single assets in a multi-asset strategy]
char name[40]; // string of maximal 39 characters
strcpy(name,Asset);
strcat(name,":");
strcat(name,Algo);
var equity = EquityShort+EquityLong;
if(equity != 0) plot(name,equity,NEW|AVG,BLUE);
\end{lstlisting}

\begin{lstlisting}[caption=Set up strategy parameters from a .ini file at the start]
function run()
{
	static var Parameter1 = 0, Parameter2 = 0;
	if(is(INITRUN)) { // read the parameters only in the first run
		string setup = file_content("Strategy\\mysetup.ini");
		Parameter1 = strvar(setup,"Parameter1");
		Parameter2 = strvar(setup,"Parameter2");
	}
}

// mysetup.ini is a plain text file that contains
// the parameter values in a format like this:
Parameter1 = 123
Parameter2 = 456
\end{lstlisting}
\begin{lstlisting}[caption=Check every minute in Trade mode if an .ini file was modified]
var Parameter1 = 0, Parameter2 = 0;

function tock() // run once per minute
{
	static int LastDate = 0;
	if(LastDate && !Trade) return; // already updated
	int NewDate = file_date("Strategy\\mysetup.ini");
	if(LastDate < NewDate) {
		LastDate = NewDate; // file was modified: update new parameters
		string setup = file_content("Strategy\\mysetup.ini");
		Parameter1 = strvar(setup,"Parameter1");
		Parameter2 = strvar(setup,"Parameter2");
	}
}
\end{lstlisting}

\begin{lstlisting}[caption=Trade multiple strategies and assets in a single script]
function run()
{
	BarPeriod = 240;
	StartDate = 2010;
	set(TICKS); // set relevant variables and flags before calling asset()
	
	// call different strategy functions with different assets
	asset("EUR/USD");
	tradeLowpass();
	tradeFisher();
	
	asset("GBP/USD");
	tradeAverage();
	
	asset("SPX500");
	tradeBollinger();  
}
\end{lstlisting}

\begin{lstlisting}[caption=Update price history of many assets]
function run()
{
	NumYears = 2;
	while(loop("AUD/USD","EUR/USD","GBP/USD","GER30","NAS100",
	"SPX500","UK100","US30","USD/CAD","USD/CHF","USD/JPY",
	"XAG/USD","XAU/USD"))
	{
		assetHistory(Loop1,1);
	}
}
\end{lstlisting}

\begin{lstlisting}[caption=Portfolio strategy with 3 assets and 3 trade functions]

function tradeTrendLong()
{
	algo("TRL");
	...
}

function tradeTrendShort()
{
	algo("TRS");
	...
}

function tradeBollinger()
{
	algo("BOL");
	...
}

function tradeFunc(); // empty function pointer

function run()
{
	while(loop(
	"EUR/USD",
	"USD/CHF",
	"GBP/USD")) // loop through 3 assets
	while(tradeFunc = loop(
	tradeTrendLong,
	tradeTrendShort,
	tradeBollinger)) // and 3 different trade algorithms
	{
		asset(Loop1); // select asset
		tradeFunc();  // call the trade function
	}
}
\end{lstlisting}

\subsection{Plotting}

\begin{lstlisting}[caption=Plot Triangle above Candle Patterns]
function run()
{
	set(PLOTNOW);
	PlotBars = 300;
	PlotScale = 8;
	if(CDLDoji())
	plot("Doji",1.002*priceHigh(),TRIANGLE4,BLUE);
	if(CDLHikkake() > 0)
	plot("Hikkake+",0.998*priceLow(),TRIANGLE,GREEN);
	if(CDLHikkake() < 0)
	plot("Hikkake-",1.002*priceHigh(),TRIANGLE4,RED);
}
\end{lstlisting}


\subsection{Systems}
\href{http://www.opserver.de/ubb7/ubbthreads.php?ubb=showflat&Number=461642#Post461642}{YenTrader System}
\begin{lstlisting}[caption=YenTrader System]

bool UseEquityFilter = false;

function TradeYenTrader() 
{
vars Price = series(price());
vars SMASlow = series(SMA(Price, optimize(100, 5, 200, 5)));
vars SMAFast = series(SMA(Price, optimize(50, 5, 200, 5)));

Stop = optimize(15, 5, 50) * PIP;
Trail = optimize(1, 1, 20) * PIP;

// Equity curve filter
var LotsBackup = Lots;

if (UseEquityFilter)
{	
vars EquityCurve = series(EquityLong+EquityShort);
vars EquityLP = series(LowPass(EquityCurve,75));
if(EquityLP[0] < LowPass(EquityLP,100) && falling(EquityLP))
Lots = -1;
else
Lots = 1;
}

if (NumOpenShort + NumOpenLong == 0) 
{
if (SMAFast[0] > SMASlow[0]) 
{
Margin = 0.5 * OptimalFLong * Capital * sqrt(max(1, Balance/Capital));
enterLong();
}
else
{
Margin = 0.5 * OptimalFShort * Capital * sqrt(max(1, Balance/Capital));
enterShort();
}
}

Lots = LotsBackup;
}

function run() 
{
set(PARAMETERS+FACTORS);

BarPeriod = 1440;
LookBack = 200;
StartDate = 20010101;
EndDate = 20160601;
MonteCarlo = 1000;
Confidence = 100;
Capital = 10000;
NumWFOCycles = 10;

asset("USD/JPY");
TradeYenTrader();	
}
\end{lstlisting}


\href{http://www.opserver.de/ubb7/ubbthreads.php?ubb=showflat&Number=461560#Post461560}{Simple profit system - EUR/USD}
\begin{lstlisting}[caption=Simple profit system]

function run()
{
	
	BarPeriod = 60;	// 1 hour bars
	LookBack = 20;	
	StartDate = 2010;
	EndDate = 20160804; 	
	
	//NumWFOCycles = 6;
	//set(LOGFILE);
	//Capital = 5000;
	//Margin = 0.001* OptimalF * Capital * sqrt(1 + ProfitClosed/Capital);
	Hedge = 2;
	Lots=1;
	NumOptCycles = 2;
	NumSampleCycles = 2;
	
	
	set(PARAMETERS+FACTORS);  
	var	StopL = optimize(3,1,10) * ATRS(optimize(3,1,10));
	var TakeProfitL=optimize(3,1,10) * ATRS(optimize(3,1,10));	
	
	var 	StopS = optimize(3,1,10) * ATRS(optimize(3,1,10));
	var 	TakeProfitS=optimize(3,1,10) * ATRS(optimize(3,1,10));	
	
	if (hour() == 13){
		Stop=StopL;
		TakeProfit=TakeProfitL;
		enterLong();
	}
	
	if (hour() == 21 ){
		Stop=StopL;
		TakeProfit=TakeProfitL;
		enterShort();
	}
	
}

\end{lstlisting}

\href{http://www.opserver.de/ubb7/ubbthreads.php?ubb=showflat&Number=460866#Post460866}{Profitable System  - EUR/USD}
\begin{lstlisting}[caption=Profitable System]
function run()
{
	NumCores = -2;		// use multiple cores (Zorro S only)
	BarPeriod = 60;	// 1 hour bars
	LookBack = 1240;	// needed for Fisher()
	StartDate = 2010;
	EndDate = 2015; 	// fixed simulation period
	
	set(PARAMETERS);  // generate and use optimized parameters and factors
	
	Stop = 5 * ATR(90);
	TakeProfit=4* ATR(30);	
	
	var nWI_TOP=optimize(30,5,35);
	var nWI_DOWN=optimize(70,60,100);
	
	vars aOpen = series(price());
	vars aAlma = series(ALMA(aOpen,
	optimize(8.61,8,30),
	optimize(6.15,6,19),
	0.8));
	
	vars MMI_Raw = series(MMI(aOpen,optimize(300,100,500)));
	vars MMI_Smooth = series(LowPass(MMI_Raw,500));
	
	vars aEma1 = series(EMA(aOpen,optimize(12.6,12,100)));
	
	vars aWI1 = series(WillR(optimize(19.9,18,49)));
	
	
	if (rising(MMI_Smooth)  &&  aWI1[0] > nWI_TOP*-1 && rising(aAlma) &&rising(aEma1)){
		enterLong();
	}else if ( falling(MMI_Smooth) && aWI1[0] < nWI_DOWN*-1 && falling(aAlma)&&falling(aEma1)){
	enterShort();
}


}
\end{lstlisting}


\href{http://www.opserver.de/ubb7/ubbthreads.php?ubb=showflat&Number=460169#Post460169}{Mean Variance Optimization}
\begin{lstlisting}[caption=Mean Variance Optimization System]
//////////////////////////////////////////////////////
// Mean Variance Optimization ////////////////////////
//////////////////////////////////////////////////////

//#define DAYS			252 	// 1 year
#define DAYS			6*22 // 6 Months
//#define DAYS			4*22 // 4 Months
//#define DAYS			2*22 // 2 Months

#define WEIGHTCAP 	.25 // Cap 0.15 - 0.5 Range
#define NN				50	 	// max number of assets
#define LEVERAGE 		4	// 1:4 leverage

//////////////////////////////////////////////////////

function run()
{
	BarPeriod = 1440;
	LookBack = DAYS;
	NumYears = 7;
	set(PRELOAD); // allow extremely long lookback period	
	set(LOGFILE);
	Verbose = 0;
	set(watch);
	
	//	AssetList = "ETF2016-OK.csv";
	AssetList = "AssetsZ8.csv";
	
	string		Names[NN];
	string 		Symbols[NN]; // Store the ISIN Code
	vars			Returns[NN];
	var			Means[NN];
	var			Covariances[NN][NN];
	var			Weights[NN];
	static int 	OldLots[NN];
	
	var TotalCapital = slider(1,1000,1000,50000,"Capital","Total capital to distribute");
	var VFactor = slider(2, 10 ,0, 100,"Risk","Variance factor");
	
	int N = 0;
	while(Names[N] = loop(Assets))
	{
		
		if(is(INITRUN) && strstr(Names[N], "#")== NULL) {
			assetHistory(Names[N], FROM_YAHOO);
			Symbols[N] = Symbol; // Store the isin code for quick referenze	
		}	
		
		if(strstr(Names[N], "#")== NULL && is(RUNNING)) {
			asset(Names[N]);
			Returns[N] = series((priceClose(0)-priceClose(1))/priceClose(1));
		}
		if(strstr(Names[N], "#")!= NULL && is(RUNNING)) Returns[N] = series(0);
		
		if(N++ >= NN) break;
	}
	
	if(tdm() == 1 && !is(LOOKBACK)){ 
		int i,j;
		for(i=0; i<N; i++) {
			Means[i] = Moment(Returns[i],LookBack,1);
			for(j=0; j<N; j++)
			Covariances[N*i+j] = Covariance(Returns[i],Returns[j],LookBack);	
		}
		var BestVariance = markowitz(Covariances, Means, N, WEIGHTCAP);
		var MinVariance = markowitzReturn(0,0);
		markowitzReturn(Weights,MinVariance+VFactor/100.*(BestVariance-MinVariance));
		
		// 		change the portfolio composition according to new weights		
		for(i=0; i<N; i++)
		if (strstr(Names[i], "#")== NULL){
			asset(Names[i]);
			MarginCost = priceClose()/LEVERAGE;
			int NewLots = TotalCapital*Weights[i]/MarginCost;
			if(NewLots > OldLots[i])	
			enterLong(NewLots-OldLots[i]);
			else if(NewLots < OldLots[i]) exitLong(0,0,OldLots[i]-NewLots);
			//					printf("\n%s - %s:  OldLots: %d NewLots: %d %.0f$",Names[i],Symbols[i], OldLots[i], NewLots );
			OldLots[i] = NewLots;
		}
	}
}

\end{lstlisting}

\href{http://www.opserver.de/ubb7/ubbthreads.php?ubb=showflat&Number=459925#Post459925}{Intraday seasonality in FX market}
\href{http://papers.ssrn.com/sol3/papers.cfm?abstract_id=2613592}{Based on this paper}
\begin{lstlisting}[caption=Intraday seasonality in FX market System]
var session1TZ = WET;
var session1Start = 8;
var session1End = 16;
var session2TZ = ET;
var session2Start = 11; // = 16 WET
var session2End = 17;

function tradeIS()
{
if(dow() >= 1 && dow() <= 5) {
if(NumOpenShort == 0 && lhour(session1TZ) == session1Start)
enterShort();
if(NumOpenShort > 0 && lhour(session1TZ) >= session1End)
exitShort();
if(NumOpenLong == 0 && lhour(session2TZ) == session2Start)
enterLong();	
if(NumOpenLong > 0 && lhour(session2TZ) >= session2End)
exitLong();
}
}

function run()
{
StartDate = 2004;
UnstablePeriod = 0;
LookBack = 0;
BarPeriod = 1;
while(asset(loop("EUR/USD")))
tradeIS();
set(LOGFILE);
}

\end{lstlisting}

\href{http://www.opserver.de/ubb7/ubbthreads.php?ubb=showflat&Number=452807&page=1}{One Night Stand System}
 
\begin{lstlisting}[caption=One Night Stand System]

function tradeOneNightStand() {
	
	vars Price = series(price());
	vars SMA10 = series(SMA(Price, 10));
	vars SMA40 = series(SMA(Price, 40));
	
	//Stop = 90 * PIP;
	
	var BuyStop,SellStop;
	
	BuyStop = HH(10) + 1*PIP;
	SellStop = LL(10) - 1*PIP;
	
	if (dow() == 5 && NumOpenLong == 0 && NumPendingLong == 0 && SMA10[0] > SMA40[0]) 
	enterLong(0,BuyStop);
	else if (dow() == 5 && NumOpenShort == 0 && NumPendingShort == 0 && SMA10[0] < SMA40[0])
	enterShort(0,SellStop);			
	
	if (dow() != 5 && dow() != 6 && dow() != 7) {
		exitLong();
		exitShort();
	}
	
}


function run() {
	
	BarPeriod = 1440;
	
	while(asset(loop("USD/CHF", "USD/JPY" , "GBP/USD", "EUR/USD")))
	tradeOneNightStand();
	
}
\end{lstlisting}

\begin{lstlisting}[caption= Portfolio trading Trend  Counter Trend and Huck Trend]
// Portfolio trading: Trend, Counter Trend and Huck Trend ///////////////////

function tradeTrend()
{
	TimeFrame = 1;
	var *Price = series(price());
	var *Trend = series(LowPass(Price,optimize(250,100,1000)));
	Stop = optimize(2,1,10) * ATR(100);
	
	if(strstr(Algo,":L") and valley(Trend)) 
	enterLong();
	else if(strstr(Algo,":S") and peak(Trend)) 
	enterShort();
}

function tradeCounterTrend()
{
	var *Price = series(price());
	var Threshold = optimize(1.0,0.5,2,0.1);
	var *DomPeriod = series(DominantPeriod(Price,30));
	var LowPeriod = LowPass(DomPeriod,500);
	var *HP = series(HighPass(Price,LowPeriod*optimize(1,0.5,2)));
	var *Signal = series(Fisher(HP,500));
	Stop = optimize(2,1,10) * ATR(100);
	
	if(strstr(Algo,":L") and crossUnder(Signal,-Threshold)) 
	enterLong(); 
	else if(strstr(Algo,":S") and crossOver(Signal,Threshold)) 
	enterShort();
}

function HuckTrend()
{
	
	TimeFrame = 1;
	var *Price = series(price());
	var *LP5 = series(LowPass(Price,5));
	var *LP10 = series(LowPass(Price,optimize(10,6,20))); 
	var *RSI10 = series(RSI(Price,10)); 
	Stop = optimize(5,1,10)*ATR(30);
	int crossed = SkillLong[0];
	int Delay = 3;
	
	
	if(crossOver(LP5,LP10))
	crossed = Delay;
	else if(crossUnder(LP5,LP10))
	crossed = -Delay;
	
	if(strstr(Algo,":L") and (crossed > 0 && crossOver(RSI10,50))) {
		enterLong();
		crossed = 0;
	} else if(strstr(Algo,":S") and (crossed < 0 && crossUnder(RSI10,50))) {
	enterShort();
	crossed = 0;
} else
SkillLong[0] -= sign(crossed);
}

function run()
{
	set(PARAMETERS+FACTORS+LOGFILE); // use optimized parameters and reinvestment factors
	BarPeriod = 240;	// 4 hour bars
	LookBack = 500;	// needed for Fisher()
	NumWFOCycles = 8; // activate WFO
	NumBarCycles = 4;	// 4 times oversampling
	
	if(ReTrain) {
		UpdateDays = 30;	// reload new price data from the server every 30 days
		SelectWFO = -1;	// select the last cycle for re-optimization
	}
	
	// portfolio loop
	while(asset(loop("EUR/USD","USD/CHF")))
	while(algo(loop("TRND:L","TRND:S","CNTR:L","CNTR:S","HuckTrend:L","HuckTrend:S")))
	{
		// set up the optimal margin
		if(Train)
		Lots = 1;
		else if(strstr(Algo,":L") and OptimalFLong > 0) {
			Lots = 1;			 
			Margin = clamp((WinLong-LossLong) * OptimalFLong/2, 50, 10000);
		} else if(strstr(Algo,":S") and OptimalFShort > 0) {
		Lots = 1;
		Margin = clamp((WinShort-LossShort) * OptimalFShort/2, 50, 10000);
	} else
	Lots = 0; // switch off trading
	
	if(strstr(Algo,"TRND")) 
	tradeTrend();
	else if(strstr(Algo,"CNTR")) 
	tradeCounterTrend();
	else if(strstr(Algo,"HuckTrend")) 
	HuckTrend();	
	
}

PlotWidth = 1000;
PlotHeight1 = 320;
}
\end{lstlisting}


\begin{lstlisting}[caption=System]

// http://www.opserver.de/ubb7/ubbthreads.php?ubb=showflat&Number=414386&page=1
function run(){
	
	set(LOGFILE);
	
	int FastPeriod = 8;
	int SlowPeriod = 21;
	int SignalPeriod = 9;
	
	BarPeriod = 15;
	vars Close = series(priceClose());
	TimeFrame = 1;
	
	MACD(Close,FastPeriod,SlowPeriod,SignalPeriod);
	vars MainLine = series(rMACD);
	vars SignalLine = series(rMACDSignal);
	vars Hist = series(rMACDHist);
	
	MACD(Close,FastPeriod*2,SlowPeriod*2,SignalPeriod*2);
	vars MainLine30 = series(rMACD);
	vars SignalLine30 = series(rMACDSignal);
	vars Hist30 = series(rMACDHist);
	
	MACD(Close,FastPeriod*4,SlowPeriod*4,SignalPeriod*4);
	vars MainLine60 = series(rMACD);
	vars SignalLine60 = series(rMACDSignal);
	vars Hist60 = series(rMACDHist);
	
	MACD(Close,FastPeriod*16,SlowPeriod*16,SignalPeriod*16);
	vars MainLine240 = series(rMACD);
	vars SignalLine240 = series(rMACDSignal);
	vars Hist240 = series(rMACDHist);
	
	
	
	if(crossOver(MainLine,SignalLine) && MainLine30[0] > SignalLine30[0]
	&& MainLine30[0] > SignalLine60[0] && MainLine240[0] > SignalLine240[0]){
		
		
		exitShort();
		enterLong();
	}
	
	if(crossUnder(MainLine,SignalLine) && MainLine30[0] < SignalLine30[0]
	&& MainLine30[0] < SignalLine60[0] && MainLine240[0] < SignalLine240[0]){
		
		
		exitLong();
		enterShort();
	}
	
	
}
\end{lstlisting}

\href{http://www.opserver.de/ubb7/ubbthreads.php?ubb=showflat&Number=416143&Searchpage=7&Main=49677&Words=logfile&Search=true#Post416143}{System}

\begin{lstlisting}[caption=System]
// 
function run ()


{
	
	StartDate = 20110601;
	EndDate = 20130120;
	BarPeriod = 1440;
	
	Stop = 50*PIP;
	Trail = 40*PIP;
	TrailLock = 1;
	
	
	vars day_low = series(priceLow());
	vars day_high = series(priceHigh());
	vars day_close = series(priceClose());
	vars EMA50 = series(EMA(day_close, 50));
	
	
	
	
	if (day_close[0] < day_low[1] && day_close[0] < EMA50[1])
	enterShort();
	
	
	plot ("EMA50", EMA50[0], 0, RED);
	
}
\end{lstlisting}

\begin{lstlisting}[caption=System]
// Build 0004

// #define ASSETLOOP "USD/CHF", "EUR/USD", "GBP/USD", "USD/CAD", "AUD/USD", "USD/JPY", "XAU/USD", "XAG/USD", "NAS100", "SPX500", "GER30", "US30", "UK100" //FOREX SET
#define ASSETLOOP "EUR/USD", "GBP/USD", "USD/CAD", "AUD/USD", "USD/JPY", "XAU/USD", "XAG/USD" // No Stock
//#define ASSETLOOP "USD/CHF", "EUR/USD","USD/JPY","XAU/USD","SPX500"//MIN FOREX SET
//#define ASSETLOOP "EUR/USD" //test asset

//---- Global VAR ----
int myCapital = 0;
var myMargin = 0;
var comp = 0;
//--------------------

function setSlider()
{
	myCapital = slider(1,2500,0,25000,"Capital"," Initial Capital"); //used for compounding calculation
	myMargin = slider(2,50,0,500,"Margin"," Initial Margin"); // fixed or initial Margin 
	comp= slider(3,0,0,1,"Comp.", "0=Fixed Margin 1=Compound Margin");
}

function checkEquity()
{
	// equity curve trading: switch to phantom mode when the equity 
	// curve goes down and is below its own lowpass filtered value
	
	if(Train) { Lots = 1; return; } // no phantom trades in training mode
	vars EquityCurve = series(ProfitClosed+ProfitOpen);
	vars EquityLP = series(LowPass(EquityCurve,10));
	if(EquityLP[0] < LowPass(EquityLP,100) && falling(EquityLP))
	Lots = -1; // drawdown -> phantom trading
	else
	Lots = 1; // profitable -> normal trading
}

function enterFSLong()
{
	if(comp == 1 && !is(TRAINMODE)) {
		Margin = OptimalFLong * myMargin * sqrt(1 + max(0,(WinLong-LossLong)/myCapital));
		enterLong();
	} else {
	Margin=myMargin;
	enterLong();
}
}

function enterFSShort()
{
	if(comp == 1 && !is(TRAINMODE)) {
		Margin = OptimalFShort * myMargin * sqrt(1 + max(0,(WinShort-LossShort)/myCapital));
		enterLong();
	}	else {
	Margin=myMargin;
	enterLong();	
}
}

function CLSTR()
{
	TimeFrame=24;
	
	Stop = 2*ATR(14);
	Trail = Stop;
	TrailLock = 10;
	
	checkEquity();	
	
	var dayL = optimize(40,10,80);
	var dayS = optimize(40,10,80);
	
	if (priceHigh() >= HH(dayL)) enterFSLong();
	if (priceLow() <= LL(dayS)) enterFSShort();	
}

function CNTR()
{
	TimeFrame = 4;
	Stop = optimize(4,2,8) * ATR(100);
	Trail = 4*ATR(100);
	
	vars Price = series(price());
	vars Filtered = series(BandPass(Price,optimize(30,20,40),0.5));
	vars Signal = series(Fisher(Filtered,500));
	var Threshold = optimize(1,0.5,1.5,0.1);
	
	checkEquity();
	
	if(crossUnder(Signal,-Threshold)) enterFSLong();
	else if(crossOver(Signal,Threshold)) enterFSShort();	
}

function TRND()
{
	TimeFrame = 1;
	Stop = optimize(4,2,8) * ATR(100);
	Trail = 0;	
	
	vars Price = series(price());
	vars Trend = series(LowPass(Price,optimize(500,300,800)));
	
	checkEquity();
	
	if(valley(Trend))	enterFSLong();
	else if(peak(Trend)) enterFSShort();	
}

function run()
{
	set(PARAMETERS+FACTORS); // generate and use optimized parameters and factors
	
	BarPeriod = 60;	// 1 hour bars
	LookBack = 2500;	// needed for Fisher()
	NumWFOCycles = 6; // activate WFO
	
	StartDate = 2009;
	EndDate = 2014;
	Hedge=5;
	
	if(ReTrain) {
		UpdateDays = -1;	// update price data from the server 
		SelectWFO = -1;	// select the last cycle for re-optimization
		reset(FACTORS);	// don't generate factors when re-training
	}
	NumWFOCycles = 6; // activate WFO
	
	setSlider();
	
	// portfolio loop
	while(asset(loop(ASSETLOOP)))
	while(algo(loop("TRND","CNTR","CLSTR")))
	{	
		if(Algo == "TRND") 
		TRND();
		
		if(Algo == "CNTR") 
		CNTR();
		
		if(Algo == "CLSTR") 
		CLSTR();
	}
	
	PlotWidth = 700;
	PlotHeight1 = 400;
	//ColorUp = ColorDn = ColorWin = ColorLoss = 0; // don't plot candles and trades
	//set(TESTNOW+PLOTNOW);
}

\end{lstlisting}

\begin{lstlisting}[caption=Enter a trade when the RSI12 crosses]
// Zorro version
// enter a trade when the RSI12 crosses over 75 or under 25
function run()
{
	// get the RSI series
	vars Close = series(priceClose());
	vars rsi12 = series(RSI(Close,12));
	
	// set up stop / profit levels
	Stop = 200*PIP;
	TakeProfit = 200*PIP;
	
	// if rsi crosses over buy level, exit short and enter long
	if(crossOver(rsi12,75))
	reverseLong(1);
	// if rsi crosses below sell level, exit long and enter short
	if(crossUnder(rsi12,25))
	reverseShort(1);
}
\end{lstlisting}

\begin{lstlisting}[caption=First Hour Breakout System]
// Zorro version
function run()
{
	
	set(TICKS+LOGFILE);
	
	BarPeriod = 30;
	LookBack  = 3;
	Hedge = 2;
	
	
	StartDate = 20100101;
	//	EndDate   = 20100201;
	
	asset("UK100");
	
	int marketstarthour		= 8;
	int marketstartminute	= 0;
	int marketendhour			= 16;
	int marketendminute		= 30;
	
	vars hi				= series(priceHigh());
	vars lo				= series(priceLow());
	
	//Find the high low range of first hours trading
	vars enthigh		= series(MaxVal(hi,2)); 
	vars entlow 		= series(MinVal(lo,2));
	vars Range			= series(enthigh[0]-entlow[0]);
	
	
	
	
	
	//	printf("\n%2.0d:%2.0d High %4.1f Low %4.1f enthigh %4.1f entlow %4.1f range %4.1f",hour(),minute(),hi[0],lo[0],enthigh[0],entlow[0],Range[0]);
	//	
	
	//	if (NumPendingTotal == 1)   // Take only one trade per day OCO one cancels other Not sure if this was part of strategy or not
	//	{
		//		for(open_trades)
		//		{
			//			if (TradeIsPending) exitTrade(ThisTrade);
			//		}
		//	} 
	
	if ((hour() == marketstarthour+1)
	and (minute() == marketstartminute))
	{
		
		//			printf("\nenthigh = %4.1f; entlow = %4.1f; Range = %4.1f; PIP = %4.1f",enthigh[0],entlow[0],Range[0],PIP);
		//			
		
		if (Range[0] < 40*PIP) 		// Don’t trade if range is higher than 40 points
		{
			//				printf("\nRange %4.1f < 40 pips, entering pending trades at enthigh %4.1f & entlow %4.1f,price %4.1f",Range[0],enthigh[0],entlow[0],priceClose());
			
			EntryTime = 15;											// Expire pending trades at 4:30
			
			enterLong(0,enthigh[0],Range[0],Range[0]);
			enterShort(0,entlow[0],Range[0],Range[0]);		//Buy on the breakout of the lowest low or the highest high of that first hour
		}
		
	}
	
	if (((hour() >= marketendhour)
	and (minute() >= marketendminute)) and (NumOpenTotal > 0))
	{
		//			printf("\nExiting Open Trades");
		exitLong();
		exitShort();
	}
}
\end{lstlisting}

\begin{lstlisting}[caption=First Hour Breakout System II]
LookBack = 100;
set(TICKS);

vars Price = series(price());

StartMarket = 800;
EndMarket = 900;

var DH = 0, DL = 0;

if(lhour(UTC) > EndMarket/100) {
	DH = dayHigh (UTC,0);
	DL = dayLow (UTC,0);
}

var Range = DH-DL;
var Close = timeOffset(UTC,0,16,30);
var Open = timeOffset(UTC,0,8,00);
var StartTrade = timeOffset(UTC,0,9,00);
var Now = timeOffset(UTC,0,0,00);

if (Now > Close)
exitLong();

if (Now > Close)
exitShort();

asset("UK100"); 

Stop = Range; 

TakeProfit = Range;

if (Range > 40*PIP)
Margin = 0;
\end{lstlisting}


\begin{lstlisting}[caption=System III - Z4 similar]

// http://www.opserver.de/ubb7/ubbthreads.php?ubb=showflat&Number=439961&Searchpage=1&Main=52773&Words=NumOpenTotal&Search=true#Post439961
// Grid trader (needs hedging - non-US accounts only)

// find all trades at a grid line
bool findTrade(var Price,var Grid,bool IsShort)
{
	for(open_trades)
	if((TradeIsShort == IsShort)
	and between(TradeEntryLimit,Price-Grid/2,Price+Grid/2))
	return true;
	return false;
}

int run() 
{
	set(TICKS);					      // Test using 1 minute Bars
	StartDate = 2012;					// Start trading from 2012
	BarPeriod = 240;              // Trade the 4 Hour chart
	Hedge = 2;                    // allow long & short trades at the same time
	EntryTime = ExitTime = 500;   // close trades that have been opened for 3 months
	asset("EUR/CHF");
	var Price;							
	var Grid = 25*PIP;            // set grid siz to 25 pips
	var Close = priceClose();		// Use the price close of bars
	
	// place pending trades at 5 grid lines 
	// above and below the current price
	for(Price = 0; Price < Close+5*Grid; Price += Grid)
	{
		// find the lowest grid line
		if(Price < Close-5*Grid) continue;
		// place not more than 200 trades		
		if(NumOpenTotal + NumPendingTotal > 200) break;
		// place short trades below the current price
		if(Price < Close and !findTrade(Price,Grid,true))
		enterShort(3,Price,20*Grid,Grid);  		
		// place long trades above the current price
		else if(Price > Close and !findTrade(Price,Grid,false))
		enterLong(3,Price,20*Grid,Grid);
	}
}
\end{lstlisting}
%%%%%%%%%%%%%%%%%%%%%%%%%%%%%%%%%%%%%%%%%
\section{Trademanagement Function}
\begin{description}
\item[TradePriceOpen] The ask price when the trade was opened. If the trade was not yet opened, it's the current price of the asset.	 
\item[TradePriceClose]
\item[TradeUnits]
\item[TradeRoll]
\item[TradeProfit]
\item[TradeEntryLimit]
\item[TradeLots]
\item[TradeIsShort]
\item[TradeIsLong] 
\end{description}
%%%%%%%%%%%%%%%%%%%%%%%%%%%%%%%%%%%%%%%%%
\section{Trading auxiallary functions}

\begin{lstlisting}[caption=TrailingStop]
int TrailingStop()
{
	// adjust the stop only when the trade is in profit
	if(TradeProfit > 0)
	// place the stop at the lowest bottom of the previous 3 candles
	TradeStopLimit = max(TradeStopLimit,LL(3));
	// plot a line to make the stop limit visible in the chart
	plot("Stop",TradeStopLimit,MINV,BLACK);
	// return 0 for checking the limits
	return 0;
}
\end{lstlisting}
\lstlistoflistings 
 



\section{Resources}
 
\vfill
\hrule
~\\
\end{multicols}

\end{document}
